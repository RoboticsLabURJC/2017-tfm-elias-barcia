\chapter{Introducción} \label{cap:introduccion}
\pagenumbering{arabic}
\setcounter{page}{1}

Visual SLAM es una técnica utilizada principalmente con robots móviles para que el robot sea capaz de autolocalizarse y generar mapas que le ayuden a esa actualización.
Basicamente se comporta como una caja negra que procesa unas imagenes captadas por una o varias cámaras y a partir de esas imágenes es capaz de obtener su posición 3D en el mundo que rodea al robot, de esta forma el robot es capaz de desplazarse en su entorno sin perderse.


\clearpage
\section{Aplicaciones}

Entre las aplicaciones de visual slam estaría el 
proyecto tango
El robot Gita de Piaggio